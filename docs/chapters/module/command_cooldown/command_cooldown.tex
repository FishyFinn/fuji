\LevelOne{command\_cooldown}\label{ch:command_cooldown}

\LevelTwo{Purpose}
This module provides:

\begin{description}
    \item [unnamed cooldown]: {per command cooldown after the \ttt{command execution}}
    \item [named cooldown]: {support to associate a named cooldown with commands.}
\end{description}

\begin{tips}{How to write regex language?}
    See:~\nameref{term:regex}
\end{tips}


\LevelThree{Example}
\begin{example}{Create a named cooldown}
    \ic{/command-cooldown create example 3000}
\end{example}

\begin{example}{Test a named cooldown}
    \ic{/command-cooldown test example <player> --onFailed "say false \ph{fuji:command\_cooldown\_left\_time 1}/\ph{fuji:command\_cooldown\_left\_usage 1}" say true}
\end{example}

\begin{example}{Reset a named cooldown for a player}
    Note that this will only reset the \ttt{timestamp} associated with the player, the \ttt{usage} associated with the player will not be reset.\\
    \ic{/command-cooldown reset example <player>}
\end{example}

\begin{example}{Create a named cooldown with 3 max usage and 15 sec cooldown}
    \ic{/command-cooldown create example 15000 --maxUsage 3}
\end{example}

\begin{example}{Create a named global cooldown for all players}
    A \ttt{named global cooldown} means that, all players shares the same cooldown, instead of \ttt{per-player}.\\
    \ic{/command-cooldown create example 3000 --global true}
\end{example}

\begin{example}{Create a non-persistent named cooldown}
    A \ttt{non-persistent named cooldown} means that, the \ttt{timestamp} associated with a player will not be persisted into the storage.
    That's to say, a server restart will forget all \ttt{timestamp}, but the \ttt{usage} associated with a player will always be persisted.\\
    \ic{/command-cooldown create example 999999999999 --persistent false}\\
    Taken this example, it means that each time the server restarted, the cooldown will be available only once.
\end{example}
