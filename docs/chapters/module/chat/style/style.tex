\LevelThree{style}
\LevelFour{Purpose}
Customize global chat style, and also allow players to use \ic{/chat style} to set their per-player chat style.

\LevelFour{Feature}
\begin{enumerate}
    \item You can use \ttt{mini-message language} to define complex format.
    \item Besides the \ttt{server chat format}, each player can also set their \ttt{per-player chat format}.
    \item This module doesn't \tbf{break} the vanilla chat events, so it can work with other chat related mods.
\end{enumerate}

\begin{tips}{Write complex style using mini-language}
    You can use \ttt{mini-language} to write complex text.\\
    \\
    See more:
    \begin{enumerate}
        \item \url{https://docs.advntr.dev/minimessage/format.html}
        \item \url{https://placeholders.pb4.eu/user/quicktext}
    \end{enumerate}
\end{tips}

\LevelFour{Configuration}
\begin{Configuration}

    \item[style]{
        The style for chat message.
    }

\end{Configuration}

\begin{example}{Set prefix and suffix for players}
    Luckperms is required to set \ttt{prefix} and \ttt{suffix}. \\
    \\
    After you installed \ttt{luckperms} mod, just issue \ic{/lp group default meta setprefix <yellow>[awesome]} to assign prefix. \\
    Don't forget to insert \ph{fuji:player\_prefix} and \ph{fuji:player\_suffix} in \ttt{sender} option in configuration file, and issue \ic{/fuji reload}
\end{example}

\LevelFour{Example}
\begin{example}{Set per-player chat style}
    \ic{/chat style set prefix + \%message\% + suffix}
\end{example}

\begin{example}{Reset per-player chat style}
    \ic{/chat style reset}
\end{example}

\begin{example}{Use chat stripe module to control the style tags usage}
    By default, the \ttt{chat style} module allows any player to use any \ttt{style tags} in chat message, including \str{<click>} tag.\\
    See~\nameref{ch:chat_stripe}
\end{example}
