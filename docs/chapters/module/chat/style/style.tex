\LevelThree{style}
Customize the server chat format and per-player chat format.

\LevelFour{Command}
\LevelFive{/chat}

\LevelFour{Feature}
\begin{enumerate}
    \item You can create your own \ttt{regex transformer} to replace \ttt{chat message} sent by players.
    \item You can insert any \ttt{placeholder} like \ph{world:name}
    \item {You can insert player's prefix and suffix. Just insert \ph{fuji:player\_prefix} and \ph{fuji:player\_suffix}.}
    \item You can insert \str{item}, \str{inv} and \str{ender} to display your item, inventory and ender-chest.
    \item You can insert \str{Steve} to mention another player named Steve.
    \item You can insert \str{pos} to show the position.
    \item You can use \ttt{markdown language} to define simple format.
    \item You can use \ttt{mini-message language} to define complex format.
    \item Besides the \ttt{server chat format}, each player can also set their per-player chat format.
    \item This module doesn't \tbf{cancel} the vanilla chat events, so it can work with other chat related mods.
\end{enumerate}

\begin{tips}{Write complex style using mini-language}
    You can use \ttt{mini-language} to write complex text.\\
    \\
    See more:
    \begin{enumerate}
        \item \url{https://docs.advntr.dev/minimessage/format.html}
        \item \url{https://placeholders.pb4.eu/user/quicktext}
    \end{enumerate}
\end{tips}

\LevelFour{Placeholder}
\begin{description}
    \item[\ph{fuji:pos}] player current position
    \item[\ph{fuji:player\_prefix}] player prefix
    \item[\ph{fuji:player\_suffix}] player suffix
\end{description}

\begin{note}{Set prefix and suffix for players}
    Luckperms is required to set \ttt{prefix} and \ttt{suffix}. \\
    \\
    After you installed \ttt{luckperms} mod, just issue \ic{/lp group default meta setprefix <yellow>[awesome]} to assign prefix. \\
    Don't forget to change the format in chat module, and issue \ic{/fuji reload}
\end{note}

\LevelFour{Configuration}
\begin{Configuration}
    \item[format]{
        The server chat format for all players.
    }

    \item[rewrite]{
        The \ttt{regex language} list used to \ttt{rewrite} the player chat message.\\
        You can use regex language to transform player's chat input (only chat message, no command usage).
    }

    \item[mention\_player]{
        \label{itm:mention_player}
        If you insert \tbf{Steve} in chat message, then the player named Steve will get audio mention.

        \begin{NestedList}
            \item[sound]{
                The type of \tbf{sound} used to notify the mentioned player.
                \begin{tips}{Query all identifiers of sound}
                    \ic{/playsound \ldots}
                \end{tips}
            }

            \item[volume]
            \item[pitch]
            \item[repeat\_count]{
                The sound repeat count.
            }

            \item[interval\_ms]{
                The interval between each repeat.
            }
        \end{NestedList}

        \item[spy]{
            Log chat information into server-console.
        }

    }
\end{Configuration}
