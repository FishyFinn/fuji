\LevelOne{placeholder}
\LevelTwo{Purpose}
This module provides more \ttt{placeholder} for \ttt{Text Placeholder API} mod.

\LevelTwo{Command}
\LevelThree{/placeholder}

\LevelTwo{Placeholder}
\begin{description}
    \item [\ph{fuji:player\_mined}] sum of mined blocks of a player
    \item[\ph{fuji:server\_mined}] sum of mined blocks of a server
    \item[\ph{fuji:player\_placed}] sum of placed blocks of a player
    \item[\ph{fuji:server\_placed}] sum of placed blocks of a server
    \item[\ph{fuji:player\_killed}] sum of killed entities of a player
    \item[\ph{fuji:server\_killed}] sum of killed entities of a server
    \item[\ph{fuji:player\_moved}] sum of moved distance of a player
    \item[\ph{fuji:server\_moved}] sum of moved distance of a server
    \item[\ph{fuji:player\_playtime}] sum of playtime of a player
    \item[\ph{fuji:server\_playtime}] sum of playtime of a server
    \item[\ph{fuji:health\_bar}] the health bar of a player
    \item[\ph{fuji:rotate hello}] rotate the string \tbf{hello}
    \item[\ph{fuji:has\_permission}] check luckperms permission
    \item[\ph{fuji:has\_meta}] get luckperms meta
    \item[\ph{fuji:random\_player}] get a random online player
    \item[\ph{fuji:random 1 5}] get a random number from 1 to 5
    \item[\ph{fuji:escape}] {escape a placeholder from the parser.
    An optional number argument is used as the levels to escape.}
    \item[\ph{fuji:protect}] {protect a string from the parser.}
    \item[\ph{fuji:date}] {get current date.

    An optional string argument is used to set the \tbf{date formatter}, for example, \ph{fuji:date HH:MM}.\\
    See also: \url{https://docs.oracle.com/javase/8/docs/api/java/text/SimpleDateFormat.html}}
\end{description}

\LevelTwo{What's more?}

\begin{tips}{Use placeholder in language file}
    It's allowed to write placeholders in language file.
\end{tips}

\begin{note}{Some other mods that provide more placeholders}
    \url{https://placeholders.pb4.eu/user/mod-placeholders/}
\end{note}



