\LevelThree{for\_each}

\LevelFour{Purpose}
This module provides /foreach command.
If a command is only targeted for single player, you can use \ic{/foreach} to apply it for each player online.

\LevelFour{Command}
\LevelFive{/foreach}

\LevelFour{Example}
\begin{example}{Say hello to online players}
    \ic{/foreach say hello \ph{player:name}}
\end{example}

\begin{tips}{Escape the placeholder properly}
    If you use foreach in scheduler module, then you should escape (Write \ph{fuji:escape player:name} instead of \ph{player:name}) the placeholder.

    It's because the command-scheduler module will try to parse the placeholder, and you need to escape the placeholder, so that the placeholder can be parsed by \ic{/foreach} command.

    Here is an example about escape the foreach command in scheduler command list: \ic{/foreach give \ph{fuji:escape player:name} minecraft:diamond 16}
\end{tips}