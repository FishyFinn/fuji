\LevelOne{fuji}\label{ch:fuji}

\LevelTwo{Purpose}
This module provides the command \ic{/fuji}, which includes some operations on fuji itself.

\LevelTwo{Command}
\LevelThree{/fuji reload}
Reload all \ttt{configuration files} and all \ttt{modules}.

\begin{note}{Module itself can't be hot reloaded}
    After you \ttt{enable} or \ttt{disable} a module, you must \ttt{restart} the server.
\end{note}

\LevelThree{/fuji about}
Open a gui to display the about, including the mod version and contributor list.

\LevelThree{/fuji inspect modules}
Inspect all the enabled/disabled modules.

\LevelThree{/fuji inspect server-commands}
Inspect all the registered commands in the server.

\LevelThree{/fuji inspect fuji-commands}
Inspect all the commands registered by fuji mod.

\begin{note}{This will not show the requirement override from command permission module}
    The \ttt{requied level permision} and \ttt{required string permission} are the \ttt{default} value set by fuji.
    If you are using~\nameref{ch:command_permission} module, then this gui will not show the \ttt{overrided requirement} of a command.
\end{note}
