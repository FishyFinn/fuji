\s{world}

\ss{Purpose}
Provides a unified world management.

\ss{Concept}
What is the difference between world, dimension and dimension type ?\\
Well, in the early stage of minecraft, it only support single-dimension, which means 1 world only contains 1 dimension.\\
And now, 1 world can support multi dimension.
Sometimes, you will see world and dimension means the same thing.\\
But clearer, we say: 1 world can contains 1 or more dimension, and each dimension has its dimension type.\\
Usually, you can say a mod adds extra dimension type and create extra dimension with that dimension type instead of extra world\\
See also: \url{https://minecraft.wiki/w/Dimension_definition}\\
See also: \url{https://minecraft.wiki/w/Dimension_type}

In vanilla minecraft, 1 world contains 3 dimensions (minecraft:overworld, minecraft:the\_nether, minecraft:the\_end). You can see the dimension of a world in world/level.dat file.


dimension type is used to create dimension, there are 4 dimension type in vanilla minecraft: minecraft:overworld, minecraft:overworld\_caves, minecraft:the\_nether and minecraft:the\_end


In order to create extra dimensions of a dimension type, you need to at least exist one dimension of that dimension type.

Instead of writing data into the file world/level.dat, fuji will load the extra dimensions in game dynamically.

The file server.properties is used for the default world properties of extra dimensions

\ss{Configuration}
\sss{blacklist}
The dimensions in the blacklist will not be operated by this module.Use blacklist to avoid mis-operation.

\ss{Example}
\begin{example}{Create an extra the\_nether dimension}
    \ic{/world create my\_nether minecraft:the\_nether}
\end{example}

\begin{example}{Delete the extra dimension}
    \ic{/world delete fuji:my\_nether}
\end{example}

\begin{example}{Reset the extra dimension with random seed}
    \ic{/world reset fuji:my\_nether}
\end{example}

\begin{tips}{Specify a seed for dimension}
    \ic{/world create my\_nether minecraft:the\_nether --seed 1234567890}\\
    \ic{/world reset fuji:my\_nether --use-the-same-seed}
\end{tips}

\begin{tips}{Make a resource-world that automatically reset every day}
    You can use command-scheduler module to execute \ttt{world reset} command automatically.
\end{tips}