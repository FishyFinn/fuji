\s{world}

\ss{Purpose}
Provides a unified world management.

\ss{Command}
\sss{/world}

\ss{Concept}

\begin{example}{What is the difference between world\tcomma\ dimension and dimension type?}
    Well, in the early stage of minecraft, a \tbf{world} only support \tbf{single-dimension}, which means 1 world only contains 1 dimension.\\
    But now, 1 world supports multi dimension.
    Sometimes, you will see \tbf{world} and \tbf{dimension} means the same thing.\\
    But clearer, we say: 1 world can contain 1 or more dimension, and each dimension has its \tbf{dimension type}.\\\\
    Usually, you can say a mod adds extra dimension type and \tbf{creates an extra dimension} with that dimension type instead of \tbf{creating extra world}.\\
    See also: \url{https://minecraft.wiki/w/Dimension_definition}\\
    See also: \url{https://minecraft.wiki/w/Dimension_type}
\end{example}

\begin{tips}{The dimension and dimension types in vanilla minecraft}
    In vanilla minecraft, 1 world contains 3 dimensions:
    \begin{enumerate}
        \item minecraft:overworld
        \item minecraft:the\_nether
        \item minecraft:the\_end
    \end{enumerate}
    You can see the dimensions of a world in world/level.dat file.
    \\\\
    A dimension type is used to create dimensions, the vanilla minecraft has the following dimension type:
    \begin{enumerate}
        \item minecraft:overworld
        \item minecraft:overworld\_caves
        \item minecraft:the\_nether
        \item minecraft:the\_end
    \end{enumerate}

    The file server.properties is used for \tbf{the only and default world}.
\end{tips}

\ss{Configuration}
\sss{blacklist}
The dimensions in the blacklist will not be operated by this module.Use blacklist to avoid mis-operation.

\ss{Example}
\begin{example}{Create an extra the\_nether dimension}
    \ic{/world create my\_nether minecraft:the\_nether}
\end{example}

\begin{example}{Delete the extra dimension}
    \ic{/world delete fuji:my\_nether}
\end{example}

\begin{example}{Reset the extra dimension with random seed}
    \ic{/world reset fuji:my\_nether}
\end{example}

\begin{tips}{Specify a seed for dimension}
    \ic{/world create my\_nether minecraft:the\_nether --seed 1234567890}\\
    \ic{/world reset fuji:my\_nether --use-the-same-seed}
\end{tips}

\begin{tips}{Make a resource-world that automatically reset every day}
    You can use command-scheduler module to execute \ttt{world reset} command automatically.
\end{tips}