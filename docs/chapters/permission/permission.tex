\LevelZero{Permission}\label{ch:permission}
\LevelOne{Definition}
A \ttt{permission} is used to decide whether \ttt{a player} can do something or not.


\LevelOne{Types}
To make the discussion clearer, we define the types of \ttt{permission} as follows:
\begin{description}
    \item [level permission] A permission level is a non-negative number used in vanilla minecraft.
    The higher number means the higher authority.
    \item[string permission] Usually, a \ttt{string permission} is introduced by a \ttt{permission plugin}, such as \ttt{luckperms}.
\end{description}

\clearpage


\LevelOne{What is the permission system used by fuji?}

\LevelTwo{Explanation}
Fuji use the mojang's \ttt{vanilla permission system}, which is based on \ttt{level permission}.\\
\\
As a convention, most of the commands registered by fuji, requires \ttt{level permission} to be \ttt{0} to use.
Only a few of the commands require the \ttt{level permission} to be \ttt{4} to use.

\LevelTwo{Set a string permission for a command}
By default, fuji only use the \ttt{level permission} as \ttt{the requirement of} a command.
However, if you want to use \ttt{string permission} for a \ttt{command}, you can use \ttt{command permission module}, which is used to \ttt{override} the requirement of an existing command.

\begin{example}{Allow players to use /seed command}
    The command \ic{/seed} provided by mojang requires \ttt{level permission} to be \ttt{3} to use.
    If you want to allow players to use \ic{/seed} command, but you don't want to grant \ttt{op} for them.
    Then in this situation, you can grant the \ttt{string permission} for them: \ic{/lp group default permission set fuji.permission.seed true}, which means that: set the requirement of command \ic{/seed} to string permission \ic{fuji.permission.seed}.
\end{example}

\begin{example}{Dis-allow players to use /list command}
    The command \ic{/list} provided by mojang required \ttt{level permission} to be \ttt{0} to use.
    If you want to dis-allow players to use \ic{/list} command, but because this command requires no \ttt{string permission} to use, so it's impossible to ban it via \ttt{luckperms}.
    In this situation, you can grant a \ttt{string permission}: \ic{/lp group default permission set fuji.permission.list false} for them, which means that: set the requirement of command \ic{/list} to string permission \ttt{fuji.permission.list}.
\end{example}

\begin {example}{Unset the override of requirement of the command}
    To undo the operation in \ttt{the first example}, just issue \ic{/lp group default permission unset fuji.permission.seed}
\end{example}


\LevelOne{Reference}
\begin{enumerate}
    \item \url{https://minecraft.fandom.com/wiki/Permission\_level}
\end{enumerate}
